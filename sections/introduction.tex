\documentclass[../main.tex]{subfiles}
\graphicspath{{\subfix{../images/}}} % Images path

\begin{document}

\section{Introduction}

Welcome to NoTeX~\cite{github}, a powerful \LaTeX\ template for all your document needs. Whether you're writing notes for your next class, a side project report, or just enjoy the look of this template, NoTeX has you covered.

\begin{figure}[H]
    \centering
    \includegraphics[width=0.15\textwidth]{pencil.png}
    \caption{Importing images from the \cc{images/} folder is easy with the \cc{graphicspath} command.}
    \label{fig:notex-logo}
\end{figure}

With NoTeX, you can easily organize your content into sections, subsections, and more. The template takes care of all the formatting, so you can focus on the content itself. It also supports importing images from a separate folder as shown in figure~\ref{fig:notex-logo}, making it convenient to include graphics in your document.

To get started, simply replace this placeholder text with your own introduction. Feel free to explore the other sections and customize the template to suit your needs.\\
If you want to add more sections, it's possible to simply create a new \cc{.tex}
file in the \cc{sections/} folder and import it in the \cc{main.tex} file as
showed for these examples sections.\\

If you're new to \LaTeX, don't worry! The template is designed to be easy to use. Just look at the example and try to replicate it. If you need help, there are plenty of resources online to help you get started~\cite{overleaf}.\\

If you like this template please visit my repository on GitHub~\cite{github} and give it a star \emoji{star}.\\

\begin{cbox}
\begin{center}
	%//TODO: link to repository 
	\faIcon{github} \; \href{https://github.com/marcotallone/notex}{\bft{marcotallone/notex}}
\end{center}
\end{cbox}

Happy writing!

\end{document}
