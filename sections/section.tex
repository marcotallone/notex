\documentclass[../main.tex]{subfiles}
\graphicspath{{\subfix{../images/}}} % Images path

\begin{document}

\section{Usage Examples}

This section presents simple examples on how to use the NoTeX template and its
features. The template is designed to be easy to use and customize, so you can
focus on the content of your document. But if you don't like something mind that
it's always possible to customize every aspect by messing with the files in the
\cc{settings/} folder.\\
In the following subsections you will find examples of how to use the template's
features.\\

\subsection{Boxes}

You can use different types of boxes to highlight important information. A
simple box can be created with the \cc{cbox} environment, which stands for
``\itt{colored box}''.\\

\begin{cbox}
	This is a simple, not really colorful, box.\\
	You can use many types of environments inside it, such as lists, equations,
	images, and more.\\
	This has been made using the \cc{tcolorbox} package, therefore floats are not welcome inside these boxes and the next ones.
\end{cbox}

You can also change the color of the box by passing an optional argument to the
\cc{cbox} environment and of course use already implemented \LaTeX\ enviromnets
inside them such as equations, double columns, lists and more\dots For example, to create an orange box, you can use the following code:

\begin{center}
	\cc{\textbackslash\ begin\{cbox\}[color]}
\end{center}

\begin{cbox}[orange]
	\textcolor{background}{
		This is an orange box.
		Maybe it's a little bit more colorful, so let's add a formula:
		\begin{equation*}
			i \hbar \frac{d}{dt} \psi(t) = \mcal{\hat{H}} \psi(t)
		\end{equation*}
}
\end{cbox}

Remember to always close all the environments you open with the corresponding
end command.
Notice that you might need to change the color inside the box to make the text
more readable. You can either use the command \cc{\textbackslash\ textcolor} or pick one of the
predefined colors commands such as:

\begin{itemize}
	\item \red{\textbackslash red}
	\item  \orange{\textbackslash orange}
	\item \yellow{\textbackslash yellow}
	\item \green{\textbackslash green}
	\item \azure{\textbackslash azure}
	\item \blue{\textbackslash blue}
	\item \purple{\textbackslash purple}
\end{itemize}

\pagebreak
\subsection{Definitions}

Colored boxes are fun but sometimes, especially in a scientific document, you
need to add some definitions. To do so, the \cc{definition} environment provides
a simple way to create numbered definitions. The general syntax wants to have a
definitions block being started with:

\begin{center}
	\cc{\textbackslash begin\{definition\}[<title>][<label>]}
\end{center}

\begin{definition}[Triangle][triangle]
    A triangle is a polygon with three edges and three vertices. Triangles are classified in:
	\begin{itemize}
		\item Equilateral: all sides are equal.
		\item Isosceles: two sides are equal.
		\item Scalene: no sides are equal.
	\end{itemize}
\end{definition}

You can then reference the definition by referencing the label you gave to it,
preceded by \cc{def:}. For example look in the source code how I defined the
definition~\ref{def:triangle}.\\
It's not mandatory to give a title, neither the label, but if you have to assign a label to a non-titled definition you still need to pass one set of empty brackets \cc{\textbackslash begin\{definition\}[][label]}.

\pagebreak
\subsection{Theorems, Propositions, Lemmas, and Corollaries}

In a scientific document, you might also need to add theorems, propositions, lemmas, and corollaries. The template provides environments for each of these, which are defined as follows:

\begin{center}
	\cc{\textbackslash begin\{theorem\}[title][label]}
\end{center}

Notice that, in order to be referenced, all three of these environments share
the counter prefix \cc{th:} and hence the numbering is shared among them.\\

\begin{theorem}[Pythagorean Theorem][pythagorean]
    In a right triangle, the square of the length of the hypotenuse is equal to the sum of the squares of the lengths of the other two sides.
    \begin{equation*}
        a^2 + b^2 = c^2
    \end{equation*}
\end{theorem}

Then also:

\begin{proposition}[Proposition Name][proposition]
	This is a nice proposition, with practical implications.
\end{proposition}

\begin{lemma}[Lemma Name][lemma]
	This is a lemma, because it missed something to be a theorem.
\end{lemma}

And finally:

\begin{corollary}[Corollary Name][corollary]
	This is a corollary, whose proof is left as an exercise to the reader.
	\emoji{joy}
\end{corollary}

Of course you can always leave the title and label empty if you don't need them.

\pagebreak
\subsection{Warnings}

In some cases you might want to highlight important warnings, i.e.\ something
that you want to immediately see when reading a particular page. For this it's
possible to use the \cc{warning} environment with:

\begin{center}
	\cc{\textbackslash begin\{warning\}[color]}
\end{center}

\begin{warning}
	Be careful: in order to compile this document you need the \cc{LuaLaTeX} compiler.
\end{warning}

If you really need to alert the reader about something important, you might thing of changing the color of the warning box like this.

\begin{warning}[darkred!40!background]
	If you don't read this the world will end in 5 minutes. \emoji{scream}
	\emoji{volcano}
\end{warning}

\vspace{2cm}

\subsection{Informative Blocks}

Observations, Notes and informations can be highlighted using the \cc{info} and
\cc{blueinfo} environments:

\begin{center}
	\cc{\textbackslash begin\{info\}}
\end{center}

\begin{info}
	\bft{Note}: The \cc{LuaLaTeX} compiler is a modern version of the \cc{LaTeX} compiler that allows you to use modern fonts and features.
\end{info}

Or the \bft{\blue{blue version}}:

\begin{center}
	\cc{\textbackslash begin\{blueinfo\}}
\end{center}

\begin{blueinfo}
	\bft{Eyecatching Note}: hey, look what I can do with my hands:
	\emoji{vulcan-salute} 
\end{blueinfo}

\pagebreak
\subsection{Examples}

Examples are what you sometimes need to make your reader understand better what
you're talking about. The \cc{example} environment is designed to help you with
this. The general syntax is:

\begin{center}
	\cc{\textbackslash begin\{example\}[title][label]}
\end{center}

where, again, you might omit eihter the title or the label if you don't need them.\\
Let's see\dots an \bft{example} of this! \emoji{smile}

\begin{example}[How to use the example environment][example]
	Here is an example of how to use the example environment.\\
	You can write whatever you want here, and it will be highlighted in a box.\\
	The color it's alittle bit different from the main text so it's not confused
	with definitions, theorems or infos which looks similar.\\
	You can still use other blocks inside but you have to change the color
	manually to \cc{\gray{\textbackslash gray}} as this won't be done automatically.
	\begin{info}
		\bft{Look:} an info block inside an example block.
	\end{info}
\end{example}


\pagebreak
\subsection{Code}

We've finally arrived to the section dedicated to code, one of the most complex
one, as you might want to write code in many different ways depending on the context.\\
Before starting, however, a warning about this!

\begin{warning}
	\bft{Warning}: the code environments are based on the \cc{listings} package,
	which is not perfect and might not always work as expected. In particular,
	not all coding languages are included out of the box. I've added a few of
	the ones I usually use in the \cc{settings/codestyles.sty} file, but you might need
	to add more if you use different languages. There you also will be able to
	add keywords or change any other option relative to the code language. If you want to change something about the code blocks, then I suggest looking at the \cc{settings/code.sty} file.
\end{warning}

\subsubsection{Inline Code}

You can write \cc{inline code} as we've already seen using the
\cc{\textbackslash cc\{\}} command.

\subsubsection{Code Blocks}

A simple code block can be created using the environment:

\begin{center}
	\cc{\textbackslash begin\{code\}\{<language>\}}
\end{center}

For example:

\begin{code}{Python}
def hello(name):
    """Greets the user by name."""
    print(f"Hello, {name}!")
\end{code}

You then might want to add a title and this can be done by passing the title as a \bft{keyword argument} to the \cc{code} environment before the mandatory programming language argument:

\begin{code}
	[title = My hello name function]
	{Python}
def hello(name):
    """Greets the user by name."""
    print(f"Hello, {name}!")
\end{code}

\begin{warning}
	Just as a general warning, \byellow{make sure the first keyword argument is on the same line as the opening `\cc{[}' bracket, otherwise it won't work}.
\end{warning}

In a similar way you can also pass independently input and output arguments:

\begin{code}
	[title = My hello name function, 
	input = A name, 
	output = Hello <name>!]
	{Python}
def hello(name):
    """Greets the user by name."""
    print(f"Hello, {name}!")
\end{code}

\begin{blueinfo}
	Just be careful that if you want to pass math symbols to the input or output arguments you have to ``protext'' them to avoid strange behaviors. This can be done with the \cc{\textbackslash protect} command.
\end{blueinfo}

The previous situation for instance usually happens when writing in Pseudocode which is an actual defined language in notex but you can modify it to your liking in the \cc{codestyles.sty}~file.\\


\pagebreak
\subsection{Tables}

Tables are a common way to present data in a structured way. \LaTeX\ already
provides many tabular environments, here I just give you a nicely formatted one
for this template that you can copy and paste. You also find this template
commented in the \cc{settings/notex.sty} file under the ``Tables'' section.\\
For normal tables I usually use this style:

\begin{table}[htb]
  \renewcommand{\arraystretch}{1.5} % Row height
  \centering
  \begin{tabular}{|c|c|c|}

    % Header (different color)
    \hline
    \rowcolor{boxcolor}
    \textbf{Column1} &
    \textbf{Column2} &
    \textbf{Column3} \\ 

    % Rows
    \hline
    Row1 & 
    Row1 & 
    Row1 \\
    \hline

  \end{tabular}
  \caption{Normal table}
  \label{tab:mytablelabel}
  \renewcommand{\arraystretch}{1} % Reset row height to default
\end{table}

While for tables with a lot of text I usually use the \cc{tabularx} environment
with the \cc{X} column type. This is useful when you have a lot of text in a
column and you want it to wrap around. This table can also extend to the full textwidth Here is an example:

\renewcommand{\arraystretch}{1.5} % Row height
\begin{table}[htb]
  \centering
  \begin{tabularx}{\textwidth}{|l|X|l|}

    % Header (different color)
    \hline
    \rowcolor{boxcolor}
	\textbf{Environment} &
	\textbf{Command} &
	\textbf{Description} \\ 

    % Rows
    \hline
	Box &
	\cc{\textbackslash begin\{cbox\}[color]} &
	Creates a colored box. \\
    \hline
	Definition &
	\cc{\textbackslash begin\{definition\}[title][label]} &
	Defines something new. \\
	\hline
	Theorem &
	\cc{\textbackslash begin\{theorem\}[title][label]} &
	States a theorem. \\
	\hline
	Proposition &
	\cc{\textbackslash begin\{proposition\}[title][label]} &
	States a proposition. \\
	\hline
	Lemma &
	\cc{\textbackslash begin\{lemma\}[title][label]} &
	States a lemma. \\
	\hline
	Corollary &
	\cc{\textbackslash begin\{corollary\}[title][label]} &
	States a corollary. \\
	\hline
	Warning &
	\cc{\textbackslash begin\{warning\}[color]} &
	Highlights a warning. \\
	\hline
	Info &
	\cc{\textbackslash begin\{info\}} &
	Provides information. \\
	\hline
	Blue Info &
	\cc{\textbackslash begin\{blueinfo\}} &
	Provides informatione. \\
	\hline
	Example &
	\cc{\textbackslash begin\{example\}[title][label]} &
	Provides an example. \\
	\hline
	Code Block &
	\cc{\textbackslash begin\{code\}[title=\dots,\dots]\{language\}} &
	Formal code block. \\
	\hline

  \end{tabularx}
	\caption{Summary of NoTeX envirnments}
	\label{tab:summary-table}
\end{table}
\renewcommand{\arraystretch}{1} % Reset row height to default


% \renewcommand{\arraystretch}{1.5} % Row height
% \begin{table}[htb]

% % Settings
% \centering
% \begin{tabularx}{\textwidth}{|l|X|l|} % <--- Adjust here as needed !
% \hline
% \rowcolor{boxcolor} % Header Color

% % Header (also adjust above)
% \textbf{Environment} &
% \textbf{Command} &
% \textbf{Description} \\ 

% % Content
% \hline
% Box & \cc{\textbackslash begin\{cbox\}[color]} & Creates a colored box. \\
% \hline
% Definition & \cc{\textbackslash begin\{definition\}\{Title\}\{label\}} & Defines a new term. \\
% \hline
% Theorem & \cc{\textbackslash begin\{theorem\}\{Title\}\{label\}} & States a theorem. \\
% \hline
% Lemma & \cc{\textbackslash begin\{lemma\}\{Title\}\{label\}} & States a lemma. \\
% \hline
% Corollary & \cc{\textbackslash begin\{corollary\}\{Title\}\{label\}} & States a corollary. \\
% \hline
% Warning & \cc{\textbackslash begin\{warning\}} & Highlights a warning. \\
% \hline
% Info & \cc{\textbackslash begin\{info\}} & Provides information. \\
% \hline
% Blue Info & \cc{\textbackslash begin\{blueinfo\}} & Provides information in blue. \\
% \hline
% Example & \cc{\textbackslash begin\{example\}\{Title\}\{label\}} & Provides an example. \\
% \hline
% Code Block & \cc{\textbackslash begin\{codeblock\}} & Formal code block. \\
% \hline
% Code Box & \cc{\textbackslash begin\{codebox\}\{language\}} & Informal code box. \\
% \hline

% \end{tabularx}
% \caption{Summary of NoTeX envirnments}
% \label{tab:summary-table}
% \end{table}




\end{document}
