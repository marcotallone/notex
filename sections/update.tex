\documentclass[../main.tex]{subfiles}
\graphicspath{{\subfix{../images/}}} % Images path

\begin{document}

\section{Update: Version 2.0! \emoji{rocket}}

Hi, version 2.0 is here! This is a major update that adds the \cc{minted}
package for new coding features. In particular now text highlighting in coding
blocks is done through the 
\href{https://pygments.org/}{\cc{pygments} library \faIcon{link}}
. This means that you need to
install it in your system to use the new features. An easy way to do so is
through \cc{pip}:

\begin{cbox}
	\inline[bash]{pip install pygments}
\end{cbox}

This version is retro-compatible with the previous one, so you can still use the
previous coding environments as before (indeed the previous sections are just as
they were).\\
Additionally I've added a new theme to celebrate this new version, it's called
\inline{Tokyo}. I hope you
like it!\\
Here I will show how the new coding environments look like. These will be easier
to use as you just need to include the language between a \cc{begin} and
\cc{end} command. Check that your required language is included in the
\cc{settings/code.sty} file and you're ready to go.

\begin{python}
def main():
		print("Hello, World!")
		print("This is a Python code block.")
		print("I hope you like it!")
		print("Goodbye!")
\end{python}

The also support the usual input/output commands. Here's an example:

\begin{python}[title=Python code block,
	input={input.txt},
	output=The file extension
]
def read_extension(filename):
		return filename.split('.')[-1]
\end{python}

Pygments supports a lot of languages, like YAML:

\begin{yaml}
name: "Pippo"
surname: "Pallo"
age: 24
list:
	- item1
	- item2
	- item3

# This is a comment
nested:
	key1: value1
	key2: value2
	key3: value3
\end{yaml}

Additionally I added an \cc{\textbackslash inline\{\}} command to have inline code snippets with
correct syntax highlighting. For example, \inline[python]{print("Hello,
World!")}.

\end{document}
