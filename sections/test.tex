\documentclass[../main.tex]{subfiles}
\graphicspath{{\subfix{../images/}}} % Images path

\tcbuselibrary{raster}
\usetikzlibrary{shadings}

\newtcolorbox{testbox}[1][boxcolor]{
  breakable,              % allow the box to be split across pages
  enhanced jigsaw,        % better frame drawing
  colback   = #1,         % background color
  colframe  = #1,         % frame color
  coltext   = textcolor,  % text color
  boxrule   = 0mm,        % frame thickness
  left      = 3mm,        % left margin
  right     = 3mm,        % right margin
  top       = 3mm,        % top margin
  bottom    = 3mm,        % bottom margin
}

\newtcolorbox{testbox2}[1][boxcolor]{
  breakable,              % allow the box to be split across pages
  enhanced jigsaw,        % better frame drawing
  colback   = #1,         % background color
  colframe  = #1,         % frame color
  coltext   = textcolor,  % text color
  boxrule   = 0mm,        % frame thickness
  left      = 3mm,        % left margin
  right     = 3mm,        % right margin
  top       = 3mm,        % top margin
  bottom    = 3mm,        % bottom margin,
	fonttitle = \bfseries,
	title=My title,
	frame style={
		left color=myorange,
		right color=mypurple,
	},
	width = 0.95\textwidth,
	center,
	shadow={-1mm}{0mm}{0mm}{fill=myorange, opacity=0.95},
	shadow={-2mm}{0mm}{0mm}{fill=myorange, opacity=0.75},
	shadow={-3mm}{0mm}{0mm}{fill=myorange, opacity=0.55},
	shadow={-4mm}{0mm}{0mm}{fill=myorange, opacity=0.35},
	shadow={1mm}{0mm}{0mm}{fill=mypurple, opacity=0.95},
	shadow={2mm}{0mm}{0mm}{fill=mypurple, opacity=0.75},
	shadow={3mm}{0mm}{0mm}{fill=mypurple, opacity=0.55},
	shadow={4mm}{0mm}{0mm}{fill=mypurple, opacity=0.35},
	overlay unbroken and first={
		\fill[myorange] 
			([xshift=-0.1mm,yshift=0.1mm]title.south west) 
			rectangle 
			([xshift=1mm]title.north west);
		\fill[mypurple] 
			([xshift=-1mm,yshift=0.1mm]title.south east)
			rectangle 
			([xshift=+0.1mm]title.north east);
	},
	% leftrule = 1mm,
	% rightrule = 1mm,
  % underlay={
  %   \fill[green] 
		% ([xshift=0mm]frame.south west) 
		% rectangle 
		% ([xshift=0mm]frame.north east);
  %   % \fill[blue] (frame.south east) rectangle ([xshift=-3mm]frame.north east);
  %   % \fill[green] ([xshift=-3mm]frame.south east) rectangle ([xshift=-6mm]frame.north east);
  % }
}

% \newtcolorbox{testbox3}[1][boxcolor]{
%   breakable,              % allow the box to be split across pages
%   enhanced jigsaw,        % better frame drawing
%   colback   = #1,         % background color
%   colframe  = #1,         % frame color
%   coltext   = textcolor,  % text color
%   boxrule   = 0mm,        % frame thickness
%   left      = 3mm,        % left margin
%   right     = 3mm,        % right margin
%   top       = 3mm,        % top margin
%   bottom    = 3mm,        % bottom margin
%   overlay={
%     \draw[#1!75!background, line width=1mm] (frame.south west) rectangle (frame.north east);
%     \draw[#1!50!background, line width=1mm] ([xshift=-1mm, yshift=-1mm]frame.south west) rectangle ([xshift=1mm, yshift=1mm]frame.north east);
%   }
% }

\newtcolorbox{testbox3}[1][boxcolor]{
  breakable,              % allow the box to be split across pages
  enhanced jigsaw,        % better frame drawing
  colback   = #1,         % background color
  colframe  = #1,         % frame color
  coltext   = textcolor,  % text color
  boxrule   = 0mm,        % frame thickness
  left      = 6mm,        % left margin
  right     = 6mm,        % right margin
  top       = 6mm,        % top margin
  bottom    = 6mm,        % bottom margin
  overlay={
		% Outer
    \draw[#1!25!background,	
					line width=1.4mm, 
					rounded corners=0.5mm
				] ([xshift=0.7mm, yshift=0.7mm]frame.south west) 
				rectangle 
				([xshift=-0.7mm, yshift=-0.7mm]frame.north east);
		% Middle
    \draw[#1!50!background,
					line width=1.4mm, 
					rounded corners=0.5mm
				] ([xshift=1.7mm, yshift=1.7mm]frame.south west) 
				rectangle 
				([xshift=-1.7mm, yshift=-1.7mm]frame.north east);
		% Inner
		\draw[#1!75!background,
					line width=1.2mm, 
					rounded corners=0.5mm
				] ([xshift=2.6mm, yshift=2.6mm]frame.south west) 
				rectangle 
				([xshift=-2.6mm, yshift=-2.6mm]frame.north east);
  }
}


\newtcolorbox{testbox4}[1][boxcolor]{
  breakable,              % allow the box to be split across pages
  enhanced jigsaw,        % better frame drawing
  colback   = #1,         % background color
  colframe  = #1,         % frame color
  coltext   = textcolor,  % text color
  boxrule   = 0.5mm,        % frame thickness
  left      = 3mm,        % left margin
  right     = 3mm,        % right margin
  top       = 3mm,        % top margin
  bottom    = 3mm,        % bottom margin,
	arc				= 0mm,			% round corners
	sharp corners,
	fonttitle = \bfseries,
	bottomtitle = 0.3mm,
	toptitle = 0.5mm,
	title=My title,
	frame style={
		left color=myorange,
		right color=mypurple,
	},
	width = 0.95\textwidth,
	center,
	% leftrule = 1mm,
	% rightrule = 1mm,
	overlay={
		\fill[myorange, opacity=0.8]
		([xshift=-1mm]frame.south west) 
		rectangle 
		([xshift=0mm]frame.north west);
		\fill[myorange, opacity=0.5]
		([xshift=-2mm]frame.south west) 
		rectangle 
		([xshift=-1mm]frame.north west);
		\fill[myorange, opacity=0.3]
		([xshift=-3mm]frame.south west) 
		rectangle 
		([xshift=-2mm]frame.north west);
		\fill[mypurple, opacity=0.8]
		([xshift=0mm]frame.south east)
		rectangle 
		([xshift=1mm]frame.north east);
		\fill[mypurple, opacity=0.5]
		([xshift=1mm]frame.south east)
		rectangle 
		([xshift=2mm]frame.north east);
		\fill[mypurple, opacity=0.3]
		([xshift=2mm]frame.south east)
		rectangle 
		([xshift=3mm]frame.north east);
  }
}























\definecolor{myorange}{HTML}{E95420}
\definecolor{mypurple}{HTML}{772953}
\definecolor{warmgray}{HTML}{AEA79F}
\definecolor{warmgray}{HTML}{AEA79F}

% ------------------------------------------------------------------------------
\newtcolorbox{testwarning}[1][boxcolor]{
  breakable,              % allow the box to be split across pages
  enhanced jigsaw,        % better frame drawing
  colback   = #1,         % background color
  colframe  = yellow,     % frame color
  coltext   = textcolor,  % text color
	colbacktitle = yellow, % title background color
	opacitybacktitle = 0.5, % title background opacity
	coltitle = textcolor,  % title text color
	fonttitle = \bfseries, % title font
	title = \yellow{\faExclamationTriangle},
	% title = \yellow{\emoji{warning}},
	detach title,before upper={\tcbtitle\;\, },
  boxrule   = 0.3mm,      % frame thickness
  left      = 2mm,        % left margin
  right     = 2mm,        % right margin
  top       = 2mm,        % top margin
  bottom    = 2mm,        % bottom margin
	titlerule = 0mm,
	arc				= 0mm,			% round corners
	sharp corners,
	width = 0.95\textwidth,
	center,
	overlay={
		\fill[yellow, opacity=0.8]
		([xshift=-1mm]frame.south west) 
		rectangle 
		([xshift=0mm]frame.north west);
		\fill[yellow, opacity=0.5]
		([xshift=-2mm]frame.south west) 
		rectangle 
		([xshift=-1mm]frame.north west);
		\fill[yellow, opacity=0.3]
		([xshift=-3mm]frame.south west) 
		rectangle 
		([xshift=-2mm]frame.north west);
		\fill[yellow, opacity=0.8]
		([xshift=0mm]frame.south east)
		rectangle 
		([xshift=1mm]frame.north east);
		\fill[yellow, opacity=0.5]
		([xshift=1mm]frame.south east)
		rectangle 
		([xshift=2mm]frame.north east);
		\fill[yellow, opacity=0.3]
		([xshift=2mm]frame.south east)
		rectangle 
		([xshift=3mm]frame.north east);
  }
  % overlay   = {           % overlay warning emoji
    % \node[anchor=north west,
    %       yshift=-1mm,
    %       xshift=1mm] 
					% at (frame.north west) {\yellow{\faExclamationTriangle}};
    % % at (frame.north west) {\emoji{warning}};
  % },
}

% Define the new tcolorbox environment with xparse
\NewTColorBox{testwarning2}{O{} O{boxcolor}}{
  breakable,              % allow the box to be split across pages
  enhanced jigsaw,        % better frame drawing
  colback   = #2,         % background color
  colframe  = yellow,     % frame color
  coltext   = textcolor,  % text color
  colbacktitle = yellow,  % title background color
  opacitybacktitle = 0.5, % title background opacity
  coltitle = textcolor,   % title text color
  fonttitle = \bfseries,  % title font
  boxrule   = 0.3mm,      % frame thickness
  left      = 2mm,        % left margin
  right     = 2mm,        % right margin
  top       = 2mm,        % top margin
  bottom    = 2mm,        % bottom margin
  titlerule = 0mm,        % line below the title
  arc       = 0mm,        % arc corners radius
  sharp corners,          % sharp corners (comment to have arc corners)
  width = 0.95\textwidth, % box width
  center,                 % center the box
  title = \IfValueTF{#1}{
    \yellow{\faExclamationTriangle\, #1} % title content if title is given
  }{
    \yellow{\faExclamationTriangle}, % title content if no title is given
	before upper={\tcbtitle\;\, },
	detach title,
  },
%   \IfValueTF{#1}{}{
% 	detach title,
% 	before upper={\tcbtitle\;\, },
%   },
  overlay={               % side lines
    \fill[yellow, opacity=0.8]
    ([xshift=-1mm]frame.south west) 
    rectangle 
    ([xshift=0mm]frame.north west);
    \fill[yellow, opacity=0.5]
    ([xshift=-2mm]frame.south west) 
    rectangle 
    ([xshift=-1mm]frame.north west);
    \fill[yellow, opacity=0.3]
    ([xshift=-3mm]frame.south west) 
    rectangle 
    ([xshift=-2mm]frame.north west);
    \fill[yellow, opacity=0.8]
    ([xshift=0mm]frame.south east)
    rectangle 
    ([xshift=1mm]frame.north east);
    \fill[yellow, opacity=0.5]
    ([xshift=1mm]frame.south east)
    rectangle 
    ([xshift=2mm]frame.north east);
    \fill[yellow, opacity=0.3]
    ([xshift=2mm]frame.south east)
    rectangle 
    ([xshift=3mm]frame.north east);
  }
}

\begin{document}

\section{Test Section}

\blindtext

\begin{testwarning2}
Hey be careful!
There is a warning here. You need to pay attention for the following reasons:
\begin{itemize}
  \item This is the first reason.
  \item This is the second reason.
  \item This is the third reason.
\end{itemize}
\end{testwarning2}

\begin{testwarning2}[Warning Title]
Hey be careful!
There is a warning here. You need to pay attention for the following reasons:
\begin{itemize}
  \item This is the first reason.
  \item This is the second reason.
  \item This is the third reason.
\end{itemize}
\end{testwarning2}

\begin{testwarning2}[Warning Title][red]
Hey be careful!
There is a warning here. You need to pay attention for the following reasons:
\begin{itemize}
  \item This is the first reason.
  \item This is the second reason.
  \item This is the third reason.
\end{itemize}
\end{testwarning2}

\blindtext

\begin{testbox2}
	\blindtext
\end{testbox2}

\blindtext

\begin{testbox4}
	\blindtext
\end{testbox4}









% Simple tests... -------------------------------------------------------------


% % \begin{tcbraster}[raster columns=2, raster equal height,
% % size=small,colframe=red!50!black,colback=red!10!white,colbacktitle=red!50!white,
% % title={Box \# \thetcbrasternum}]
% % \begin{tcolorbox}First box\end{tcolorbox}
% % \begin{tcolorbox}Second box\end{tcolorbox}
% % \begin{tcolorbox}This is a box\\with a second line\end{tcolorbox}
% % \begin{tcolorbox}Another box\end{tcolorbox}
% % \begin{tcolorbox}A box again\end{tcolorbox}
% % \end{tcbraster}

% % \begin{cbox}
% % 	First box
% % \end{cbox} 


% % \begin{ebox}
% % 			Hey
% % 		\end{ebox}

% \begin{theorem}[Residual theorem]
% 	Let $f$ be \bft{holomorphic} in a domain $D$ except for isolated singularities at
% 	$a_1, a_2, \ldots, a_m$. Let $\gamma$ be a closed rectifiable curve in $D$
% 	which does not pass through any of the points $a_k$. Then
% 	\begin{equation*}
% 		\oint_\gamma f(z) \, dz = 2 \pi i \sum_{k=1}^m
% 		\text{I}(\gamma, a_k) \text{Res}(f, a_k).
% 	\end{equation*}
% 	\tcblower
% 	Here, $\text{I}(\gamma, a_k)$ denotes the winding number of $\gamma$ around
% 	$a_k$.
% \end{theorem}

% \tcbset{colback=warmgray!30!white,fonttitle=\bfseries}

% \begin{tcolorbox}[
% 	enhanced,
% 	title=My title,
% 	frame style={
% 		left color=myorange,
% 		right color=mypurple,
% 	},
% 	fuzzy shadow={0mm}{0mm}{-2mm}{2mm}{black},
%   boxrule   = 0mm,      % frame thickness
% ]
% This is a \textbf{tcolorbox}.
% \tcblower
% This is the lower part.
% \end{tcolorbox}

% \blindtext

% % preamble
% \tcbset{skin=enhanced,
% 	fonttitle=\bfseries,
% 	frame style={
% 		upper left=blue,
% 		upper right=red,
% 		lower left=yellow,
% 		lower right=green
% 	},
% 	interior style={white,opacity=0.5},
% 	segmentation style={black,dashed,opacity=0.2,line width=1pt}
% }
% \begin{tcolorbox}[title=Nice box in rainbow colors]
% With the \enquote{enhanced} skin, it is quite easy to produce fancy looking effects.
% \tcblower
% Note that this is still a \texttt{tcolorbox}.
% \end{tcolorbox}

% \blindtext

% % preamble
% \tcbset{skin=enhanced,
% 	fonttitle=\bfseries,
% 	frame style={
% 		upper left=mypurple,
% 		upper right=myorange,
% 		lower left=mypurple,
% 		lower right=myorange
% 	},
% 	interior style={boxcolor,opacity=0.7},
% 	boxrule=0mm,
% 	segmentation style={black,dashed,opacity=0.3,line width=0.1mm}
% }
% \begin{tcolorbox}[title=Nice box in rainbow colors]
% With the \enquote{enhanced} skin, it is quite easy to produce fancy looking effects.
% \tcblower
% Note that this is still a \texttt{tcolorbox}.
% \end{tcolorbox}

% \emoji{smile} \emoji{rocket}

% % ------------------------------------------------------------------------------
% \pagebreak

% \blindtext

% \tcbset{colback=warmgray!30!white,fonttitle=\bfseries}

% \begin{tcolorbox}[
% 	enhanced,
% 	title=My title,
% 	frame style={
% 		left color=myorange,
% 		right color=mypurple,
% 	},
% 	boxrule=0mm,      % frame thickness
% ]

% \blindtext

% \end{tcolorbox}

% \begin{tcolorbox}[
% 	enhanced,
% 	title=My title,
% 	frame style={
% 		left color=myorange,
% 		right color=mypurple,
% 	},
% 	% fuzzy shadow={0mm}{0mm}{-2mm}{2mm}{black},
%   boxrule   = 0mm,      % frame thickness
% ]
% This is a \textbf{tcolorbox}.
% \tcblower
% This is the lower part.
% \end{tcolorbox}



















































\end{document}
